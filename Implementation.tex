
\chapter[Implementation]{Implementation}
\graphicspath{ {images/implementation} }
This chapter details sensorial SYN, a tool that implements the software evolution comprehension approach defined in chaper X. 


\section{Platform overview}

SYN is a platform tools that allows developers to have a visual and auditive deption of an evolving system. 
This section aims to describe the tools and modules that are part of SYN. 

\subsection{SYN CLI}
SYN CLI is a command line interface that allows developers to interact with SYN. It gives to developers full control over the system. 
For example, with the command \texttt{syn project create} it is possible add a project and then, analyze it with the command \texttt{syn analyze}. 

* LIST OF AVAIL COMMANDS - APPENDIX? *


\subsection{SYN Analyzer}

SYN works with evolutionary metrics that represents the history of a system. To this aim, we developed SYN Analyzer, a Java tool on top of jGit. 
Having a languange-agnostic implementation, it can analyze every git repository written in any programming language. 

Four steps compose the analysis process: 
    \begin{enumerate}
        \item The repository is cloned.
        \item The source code of the HEAD revision is obtained. 
        \item All the files are analyzed and the metrics are obtained.
        \item If the revision has a parent, the step 2 is repeated with the parent revision.
    \end{enumerate}

As a result, starting from the HEAD revition it will go back in time until the first revision.
All the collected metric will be stored in a object, called analysis result, that can be serialized in a JavaScript Object Notation (JSON) object. 
We chose JSON because is a lightweight, easy to use and human readable format.
It is capable to analyze large repositories, as it uses a join algorithm to merge these analysis results if they are computed in parallel.

\subsection{SYN Server}
SYN Server is responsable for providing the elaborated information, given by the analysis results, in an intermediate languange between the 
front-end (SYN Debugger) and the back-end. We chose to spin up a GraphQL web server, that uses JSON as exchange language between the front-end and the back-end.
In this way, the front-end can ask exactly for the information it needs, and the back-end can send it back once they are computed. 

The computation made by the back-end, is responsable to create the view that will be shown in the front-end.
To do that, the server process a \it{view specification}, that must be given by the front-end, 
and then provides to the front-end JSON objects representing only the object that must be depicted. 
Although the information provided by the server are limitated to the view itself, it also provides debugging information if requested. 

\subsection{SYN Debugger}

SYN Debugger is a web application that allows developers to interact with SYN. It is written with React.js, a popular JavaScript framework. 
The aim of this application is to have a visual depition of the view generated by the server, plus some additional information. 
For example, it allows you to ckick on a entity and see the information that is related to it. 
The visualization is based on Babylon.js, a popular 3D library. 
SYN Debugger provides different kind of customizations to the view, such as the shape and the colors of the entities. 
All theese customizations are sent to the back-end server, through a \it{view specification} file. 

The main purpose of this application is to debug the view and explore all the possible visualization combinations of a system. 

\subsection{SYN Sonic}

....

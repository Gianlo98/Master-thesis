\chapter{Introduction}

In 1971 Dijkstra made an analogy between computer programming and art \cite{Dijkstra1971a}.
It stated that it is not essential to learn how to compose software but instead, it is necessary to develop its own style and implications. 
During the lifetime of a software system, many developers work on it, each one in their manner. This is one of the multiple reasons behind software complexity. 
Today's systems are characterized by sheer size and complexity. Software maintenance takes up the most of a system's cost. 
It is hard to quantify the impact of software maintenance on the global cost of the software. 
However, Researchers estimated it to be between 50\% and 90\% \cite{Davis1995} \cite{Sommerville1995}\cite{Erlikh2000} \cite{seacord2003}.
Many factors influence the cost of maintenance; among these, there is the understanding activity needed to perform maintenance tasks \cite{Corbi1989}. 

The comprehension of software evolution is essential for the understandability and, consequently, maintainability of systems.
However, the sheer quantity and complexity of the information generated during systems development challenge the comprehension process.

Lehman and Belady, in 1985, were the first to observe that maintaining a software system becomes a more complex activity over time. \cite{Lehman1985}
The term "software evolution" was used for the first time in their set of laws. 
One of the goals of its analysis is to identify potential defects in the system's logic or architecture. 
Visualization techniques often support software evolution analysis.
\bigbreak
Software visualization is a specialization of information visualization with a focus on software \cite{Lanza2003}. In literature, a lot of visualization techniques have been presented to support 
a complex software system's analysis.
Usually, a massive quantity of multivariate evolutionary data needs to be depicted. Several tools have been proposed in the literature to do that \cite{Merino2018a}.

Moreover, numerous techniques have been presented in the literature to facilitate program comprehension. 
The main challenge that each visualization technique has to deal with is to identify the relevant aspects to be depicted and effectively present them. 
The effectiveness of a software visualization technique could be enhanced by combining it with audio. 
The term "program auralization" was coined for this reason, and it aims to communicate information about the program in an auditory way.
Several studies were done to measure the advantages given by audio as a communication medium \cite{Alty1995}. 


In this work, our focus is an explorative visualization that depicts the evolution of a system.
We present an approach with an evolutionary design-level visualization to facilitate the comprehension of the system's history. 
Our technique models and mines efficiently large git repositories. We also offer a visualization strategy based on the concept of synesthesia 
(the production of a sense impression relating to one sense by stimulation of another sense), 
which represents the evolutionary process through an interactive visual depiction of the evolving software artifacts complemented by an auditive portrayal of the evolution. 

\section{Contribution}
We can summarize the main contributions of this work as follows:
\begin{itemize}
 \item an approach and an implementation to model the history of a large git repository.
 \item an approach and an implementation to represent the evolutionary process through an interactive visual depiction of the evolving software artifacts.
 \item an approach and an implementation to auralize evolutionary information. 
\end{itemize}

\section{Structure of the document}
This document is organized as follows
\begin{itemize}
 \item In Chapter 2, we describe the state of the art in the field of software visualization, repository mining, and software auralization. We look at evolution models, 2D and 3D visualizations. 
 \item In Chapter 3, we describe three approaches. One to mine and model the history of large git repositories, one to visualize evolutionary information of software systems and one to produce an auditive model of the evolution. 
 \item In Chapter 4, we present SYN, a software platform that implements our approach. 
 \item In Chapter 5, we validate our approach by analyzing three open-source software systems and we present the results. 
 \item In Chapter 6, we conclude the thesis with a summary and possible directions for future work.
 \end{itemize}

\chapter[Approach]{Approach}
\graphicspath{ {images/approach} }

In this chapter we will present an approach that produce a visual and auditive depiction of the evolution of a system. 
Our approach consists of two parts: 
In the first part we modelled the evolution of the software artifacts, and we enginereed a tool that implement it.
In the second part, we used the concept of synesthesia (the production of a sense impression relating to one sense by stimulation of another sense) 
to enhance the effectiveness of the visualization. 

\section{Evolutionary model}
Analyzing the evolution of a systems requires to consider numerous aspects. 
In our approach we focus on systems that resides on git repositories. 
We made this choiche because git is the most common repository management system and it also track all the changes made to the system.
As a consequence we can use it to reconstruct the history of a system.\\

In our approach, we model the evolution of repository files, therefore we model its history. 
To do that, we considered all the information that can be extracted from a git repository: files and commits.

Git uses commit to record changes on a repository. So, each commit is a snapshot of the repository at a particular point in time.
Git has the possibility to inspect every commit of a repository by using the command \texttt{git checkout}. 
In this way, we can navigate through the history of a repository, track all the files and their metrics.\\

A commit operation contains also other meta-information such as the author, the date, the message and the list of files that were updated. 
In our approach a commit is represented with a Project Version, and a file is represented with a FileHistory. 
Both a FileHistory and a ProjectVersion contains a list of FileVersions, wich represents a file at a particular point in time.

For example, if we have a file A.java that is being modified in commit c1 and c3, we will have a FileHistory A.java with 
2 FileVersions: the fileVersion of c1 and the fileVersion of c2. 
With the same fashion, ProjectVersions c1 and c3 will have the same FileVersions referred to the FileHistory A.java.

